\documentclass[12pt]{article}
\usepackage{setspace}

\usepackage{fullpage}
\usepackage{wrapfig}
\usepackage{enumerate}
\usepackage{graphicx}
\usepackage{graphics}
\usepackage{comment}
\usepackage{amsmath,amssymb,amsfonts,amsthm,amsbsy}
\usepackage{stmaryrd} %bbm,
\usepackage{verbatim}
\usepackage[authoryear]{natbib}
\usepackage{soul}
\usepackage{multirow}
\usepackage{makecell}
\usepackage{bm}
\usepackage{hyperref}
\hypersetup{colorlinks=true,
    linkcolor=blue,
    filecolor=blue,
    citecolor=black,
    urlcolor=cyan}
\usepackage{xr}

\externaldocument{lrd-r1}
\externaldocument{lrd-sec2+3-r1}

\title{Response to Reviewers}

\begin{document}
\maketitle

[Note: for the time being I'm using this for bookkeeping---it will
have to be edited before we can send it in]\\

\section{Reviewer 1 ``Litmitless RDD.pdf''}

\begin{quote}
However, the authors did not explain how this limit concept is
differently (or in the same manner) defined or interpreted from what
many people usually named as local average treatment effect
(LATE). The important concept is statistically defined but it is also
important to provide conceptual definition which would be changed or
not changed under the limit understanding.
\end{quote}

\begin{quote}
I also appreciate the motivate example but was also not sure how these
contaminated data and discrete assignment variable issues are actually
related to the presented simulation conditions or authors even tries
this or not.
\end{quote}

\begin{quote}
Furthermore, it seems that concern about discrete variable measured in
1/100s of a grade is a bit overly exaggerated concern. In social
science, many scales that have more than seven levels are often
considered as continuous variables in analysis and considering the
second decimal points actually allows pretty much continuous
variable. If we follow this standard that authors pointed out, all of
latent scores that even use the item response pattern should be
considered as discrete.
\end{quote}

\begin{quote}
In addition, providing more information on how much (proportion) of
the data were actually contaminated by social corruption would make
the argument mode valid.
\end{quote}

\begin{quote}
On page 3, the authors stated “Either circumstance calls into question
the appropriateness of limit-based methods” but did not explain why it
is so.
\end{quote}

\begin{quote}
The authors fail to reference Dong (2015) which provides a correction
for running variables that are discretized due to rounding or heaping
(such as in their motivating example).
\end{quote}

\begin{quote}
From line 45, the authors explain the organization of each subsection
by stating the limitless RDD analysis combines ideas from each
subsection. However, by the time I reach p. 12, I am almost lost why
each section was explained because the authors did not connect the
ideas closely toward the newly proposed method. Providing more guiding
sentences that tell readers why the authors discuss each section and
connecting ideas directly in a logical manner will be much appreciated
by the readers.
\end{quote}

\begin{quote}
In general, it seemed to me that the authors provided too much
background information that was not pertinent to their argument in
section 2. The first paragraph of section 3 was somewhat out of place
in the article and I suggest to drop and add sentences to remind the
readers the purpose of section 3.
\end{quote}

\begin{quote}
In section 4, the authors provide an empirical application of their
new method and compare the results to standard RD methods. However,
the details about implementation (e.g., R package or written code) are
not provided (with an exception of bandwidth calculation) and it is
not easy for readers to adopt the method even after reading the
manuscript. Like many of published manuscripts in JEBS, I strongly
recommend to providing code or more details on implementation of the
proposed method.
\end{quote}

\begin{quote}
The authors also conducted two simulations \dots Most importantly, there was no justification for the considered
conditions (e.g., sample size, effect size). These studies do not
provide sufficient information about the quality of the treatment
effect estimates themselves.
\end{quote}

\begin{quote}
The article would benefit from focusing more on the treatment effect
estimates by providing results in the form of treatment estimate bias,
95\% interval coverage, and quality of standard error estimates.
\end{quote}

\begin{quote}
Furthermore, more conditions should be included in the simulations. In
practice, standard RD studies will often report sensitivity analysis
in which the RD analysis is done using different size bandwidths, or
windows. As one of the benefits of the proposed method is the greater
power and larger sample size, the “limitless” method should be
compared to standard RD using several different bandwidths, or
windows.
\end{quote}

\begin{quote}
The authors often choose terms that are less common in the education
literature when a more appropriate term exists. For example, they use
“window” instead of “bandwidth” to describe the area surrounding the
cutoff, in which the RD analysis is conducted. Similarly, they state
that standard RD estimates the average treatment effect within a
region but fail to call that estimate the “local average treatment
effect”. In addition, the use of “level” instead of “Type I error
rate,” and the format of the number themselves, in Table 3.
\end{quote}

\begin{quote}
Last but not least, the article does not follow APA formatting at
all. The tables and figures are not properly titled or captioned. The
tables are improperly formatted, and the reference list is not in APA
format. In text citations also need to be correct to follow APA
format.
\end{quote}


\section{Editor (Dan M)}

\begin{quote}
This can be accomplished by better sign posting of what
is background and what is a specific development of your
method.
\end{quote}
\begin{quote}
\dots More generally, your actual contributions
are not clear described. For example, it is not clear if use of the
residuals for RDD is an innovation of this paper or a method being
taken from the literature.
\end{quote}
We have attempted to clarify this distinction by restricting Section
\ref{sec:review} to background and Section \ref{sec:theMethod} to new
developments. Regarding residualization, we refer (in the
Introduction) to our main
methodological contribution as a ``novel identifying assumption termed
`residual ignorability.''' and generally use the term ``residual
ignorability'' to define our method.
Although we note (e.g. \S~\ref{sec:robust-analys-covar}) that
traditional methods may be expressed in terms of residuals, we hope
that these changes make clear that we take residualization to be our
main, novel, contribution.

\begin{quote}
Language and notation should use the most common forms not
uncommon ones. For example the overbar arrow is not the common
notation for vectors the educational literature. It is more common in
the econometrics literature. Similarly, joint tests of multiple
hypotheses are not typically referred to as intersection tests. MM
estimation maybe more commonly referred to as robust estimation. If
the uncommon terminology and notation has significant advantages use
them but draw the connections to the more common forms and heuristic
understandings.
\end{quote}
(I couldn't find ``overbar arrow'' notation?) We removed reference to
the ``intersection hypothesis'' but retained reference to the
``sequential intersection-union principal.'' We referred to MM
regression as ``robust'' regression, unless the distinction between MM
and other robust models was intrinsic to our point.

\begin{quote}
I found the formal development of randomization based inference in
Section 2 useful but the distinction of the purely background material
from the development that was specific importance for your innovations
was not sufficiently clear.
\end{quote}
We moved our discussion of Fisherian randomization based inference to
Section \ref{sec:using-eqref-test}, and re-worded around our
analysis of Hurricaine Maria death tolls. We hope this clarifies its
connection with our method. We removed the discussion of Neymann-style
inference which was less directly connected to our method.

\begin{quote}
Section 2.3 is confusing. The importance of leverage and contamination
are not made clear. Contamination needs to be clearly defined. If the
window is small, the leverage from the edges doesn’t seem that
great. Is this also the case where the model is assumed nonlinear and
so failure in the approximation with a linear function also breaks
down further from the discontinuity. The use of the window to motivate
the contamination and the need for robust regression needs clarity.
\end{quote}
We re-wrote our description of sample contamination (now in
\S~\ref{sec:robustFitters}) to clarify the problem, and its connection
to bandwidth selection.
We also modified our discussion of non-robust non-linear modeling
(\S~\ref{sec:nonlinear}) to clarify the issue of leverage.

\begin{quote}
Also the discussion starts with a robustness to functional form but
the discussion transitions to robust regression and robustness to
leverage. How these are the same issue needs to be clearer than the
current discussion. The transition from describing problem with
regression to your preferred fix isn’t clear. MM estimation is one of
the key contributions of the paper but it is introduced in the middle
of a paragraph with an introductory sentence that does not signal this
a focusing on a key innovation. Moreover the language seems offhand as
if this is a method that could be considered but not one that is being
recommended.
\end{quote}
We moved the introduction of robust MM estimation to Section
\ref{theMethod} to better frame it as a novel contribution, and
clarify its crucial role in the method we are proposing.

\begin{quote}
There should also be a clearer explanation of the role of a window in
limitless RDD. In the introduction the expectation in the definition
is not limited to a window.
\end{quote}
We attempted to clarify the role of the window in
\S~\ref{sec:bandwidth} and \S~\ref{sec:bandwidthChoice}. In the
introduction we write ``The residual ignorability assumption and corresponding ATE
estimates pertain to all subjects in the window of analysis.'' linking
residual ignorability with a window from the outset \marginpar{Is this
  what he meant by ``In the introduction\dots''?}

\begin{quote}
The first paragraph of section 3 is superfluous and such superfluous
material can district or worse confuse readers. \dots
I would also be useful if there
was some explanation of the point of Section 3.1 before the section or
after it.
\end{quote}
We replaced the paragraph in question with one that explains the
purpose of each subsection of section 3.

\begin{quote}
On p. 13 line 49 the sentence “Conditioning on $e_{\theta\infty}(\mathbf{Y}_C \mid \mathbf{R})$... is a good example of
writing where the point maybe lost some readers because of the way it
is presented. The entropy in R will not be obvious to readers and they
will need to figure out what you mean.
\end{quote}
We simplified the sentence and replaced ``entropy'' with
``randomness.''

\begin{quote}
Pp 14­15, setting alpha to 0.15 to compensate for the conservatism in
the Bonferroni correct seems odd. You will reject less but you lost
the control of familywise error at the originally desired
level. Wouldn’t a less conservative method with the desired alpha
level be better?
\end{quote}
Setting alpha to a higher level will increase the power of the
specification tests to detect violations of the assumptions. We've
re-worded in \S~\ref{sec:bandwidthChoice}.
We also added citations to more powerful multiplicity adjustments in
\S~\ref{sec:balanceTesting}; however, we kept the Bonferroni procedure
in our analysis for the sake of simplicity.

\begin{quote}
In Figure 1, a vertical line at zero might be helpful.
\end{quote}
The line has been added.

\begin{quote}
In Section 5.1, it would be helpful if you explicitly defined $Y_T$.
\end{quote}
The new Section \ref{sec:levelPower} implements this suggestion.

\begin{quote}
In Table 3 the values appear to be percentages not proportion. However,
in Table 4 you present levels as proportions.
\end{quote}
In the new manuscript, we present bias, confidence interval coverage
and width, and root mean squared error. We express confidence interval
coverage as a percentage, and note it as such.

\begin{quote}
In Section 5.2 all the
generating functions are locally linear. With the appropriate window
local linear is exactly correct. Limitless is not. As generating
function with curvature might be an interesting alternative.
\end{quote}
We replaced the ``one-side'' data generating function with a
sinusoidal function, for which all the methods are misspecified.

\begin{quote}
In your inference, the fact that the same data are used to select W
and test the RDD is not considered and doesn’t seem to matter in the
simulation. Can you provide some insight for this or a reference that
discusses this?
\end{quote}
We pointed out in \S~\ref{sec:bandwidthChoice} that the specification
checks use only $R$, $Z$, and covariates, but not $Y$, and give
references that that protects outcome inference.

\begin{quote}
In the Discussion the reference to random number generation is again a
distraction and should be removed.
\end{quote}
We have removed the reference.

\end{document}