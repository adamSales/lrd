\section{Robust RDD}
Capturing the local randomization essence of the RDD calls for
a merger of RDD methods with concepts developed for RCTs. This
section selectively reviews relevant literatures.

Let $Z \in \{0,1\}$ indicate assigment to treatment ($Z=1$) as opposed to control
($Z=0$).  For an RDD one defines $Z \equiv \indicator{R< 0}$,
$\indicator{R \leq 0}$, $\indicator{R\geq 0}$ or $\indicator{R > 0}$,
depending on how intervention eligibility relates to the threshold,
where $\indicator{x}=1$ if $x$ is true and $0$ otherwise.
Let $Y$ represent the outcome of interest.
For simplicity assume non-interference, the model that
a subject's response may depend on his but not also on other subjects'
treatment assignments \citep{cox:1958,rubin:1978}.  Thus we may take each $i$
to have two potential outcomes, $y_{Ti}$ and $y_{Ci}$, at most one of which is observed, depending on whether $z_i=1$ or $0$, respectively;
 observed
responses $Y$ coincide with $ZY_{T}+(1-Z)Y_{C}$.


\subsection{The \textsc{Ancova} Model for RDDs}\label{sec:robust-analys-covar}

The classical analysis of covariance (\textsc{ancova}) model for
groups $i=1,\ldots, k$, each including subjects $j=1, \ldots, n_{i}$,
says that
$Y_{ij} = \alpha_{i} + \beta X_{ij} + \epsilon_{ij}$, where $\epsilon_{ij}
\sim \mathrm{Normal}(0, \sigma^{2})$ is independent of the continuous
covariate $X_{ij}$.
In the classical development of RDDs, \textsc{ancova} with $k=2$
groups---treated and untreated---is a leading option among statistical
models
\citep{thistlethwaite1960regression}.
A potential outcomes version of the model is
 $Y_{Ci} = \alpha_{0} + \beta R_{i} + \epsilon_{Ci}$ and
$Y_{Ti} = \alpha_{1}  + \beta R_{i} + \epsilon_{Ti}$, with
 $\epsilon_{Ci} \sim \mathrm{Normal}(0, \sigma^{2})$ and
 $\epsilon_{Ti} \sim \mathrm{Normal}(0, \sigma^{2})$.
In marked contrast to RCTs, it does
not require that $(Y_{T}, Y_{C}) \independent Z$: to the contrary, both
$Y_{C}$ and $Y_{T}$ are presumed to associate with $R$, which in turn
determines $Z$. 
Nonetheless, under this model the estimated $Z$ coefficient from the
model
\begin{equation}\label{eq:classicOLS}
Y_i=\alpha+\beta R_i+\tau Z_i+\epsilon_i
\end{equation}
fit using ordinary least squares (OLS), is unbiased for
$\alpha_{1} - \alpha_{0}$. 
This estimand is the value of  $\lim_{r\downarrow 0} \EE(Y | R=r) -
\lim_{r\uparrow 0} \EE(Y | R=r)$ and, simultaneously,
limitless estimation targets such as $\EE Y_{T}  - \EE Y_{C}$.

To restate this estimator in the manner of
\S~\ref{sec:using-eqref-test}---testing hypotheses with a test
statistic that contrasts treated and untreated subjects, after
adjusting for $R$---
onsider the hypothesis $H: Y_{T} = Y_{C} + \tau$.
Next, define ${{Y}_H} = {Y} - \tau {Z}$ (so that under $H$, $Y_H=Y_C$)
and residuals $\resid\equiv\dt[(a, b)]{\mathbf{y}_{H}}{ \mathbf{r}} = {\mathbf{y}_H} - a -
b\mathbf{r}$.
Finally, test $H$ with statistic
\begin{equation*}
t(\mathbf{Y}_H , \mathbf{Z}) =
\overline{\dt[(\hat{\alpha},\hat{\beta})]{{{Y}_H}}{ {R}}}_{Z=1} -
\overline{\dt[(\hat{\alpha},\hat{\beta})]{{{Y}_H}}{ {R}}}_{Z=0},
\end{equation*}
where $\hat{\alpha}$ and $\hat{\beta}$ are the standard OLS estimates.
Because of the structural relationship
between $R$ and $Z$, (and, unlike in
\S~\ref{sec:using-eqref-test}, regression parameters were not estimated
with a separate dataset) 
permutation tests based on this statistic are not tractable. 
However, under the parametric \textsc{ancova} model, with
conditioning on $\mathbf{R}$ rather than on $(N_{0}, N_{1},
\mathbf{Y}_{C})$ as in \S~\ref{sec:using-eqref-test},
$t(\mathbf{Y}_H , \mathbf{Z})$ is straightforwardly Normal, with
variance equal to the classical OLS variance of the coefficient on
$Z$. 
The HL estimate for $\tau$ corresponding to these statistical tests is
algebraically equal to the $Z$-coefficient from an OLS estimate of
\eqref{eq:classicOLS}.  

The model for $(Y_{T}, Y_{C})$ is not readily dispensed with, but it
may be relaxed.  OLS estimates of $\alpha_{0}$, $\alpha_{1}$ and
$\beta$ remain unbiased under non-Normality, provided
the $\epsilon$s have expectation 0 and bounded variances.  The
ordinary \textsc{ancova} standard error does not require Normality of
the $( \epsilon_{i}: i )$, either, for use in large samples, although
it does require that they have a common variance.  To test
$\EE\{ \dt[\hat\theta]{{Y_H}}{ R} | Z=1\} = \EE\{ \dt[\hat\theta]{{Y_H}}{
R} | Z=0\}$
under potential heteroskedasticity, one estimates
$\var\left\{ \overline{\dt[\hat\theta]{{Y_H}}{ R}}_{Z=1}
  -\overline{\dt[\hat\theta]{{Y_H}}{ R}}_{Z=0} \right\}$
using a sandwich or Huber-White estimator,
$\mathrm{SE}_{s}^{2} \left\{ \overline{e({Y_H}| R)}_{Z=1} -
  \overline{e({Y_H}| R)}_{Z=0} \right\}$
\citep{huber1967behavior,mackinnonWhite1985sandwichHC,longErvin2000sandwichHC,
bellmccaffrey2002sandwichSEs,pustejovskyTipton2017sandwichSEs}, %as opposed to the %$\mathrm{SE}_{p}$; then
%$\mathrm{SE}_{u} %( \overline{e({Y_H}| R)}_{Z=1} -  \overline{e({Y_H}| R)}_{Z=0} )$
%of \eqref{eq:tudef}:
and refers
\begin{equation} \label{eq:tedef}
t_{e} (\mathbf{y}_H,\mathbf{r}) =
\frac{\overline{\dt[\hat\theta]{{y_H}}{ r}}_{z=1} -
                             \overline{\dt[\hat\theta]{{y_H}}{ r}}_{z=0}}%
                           {\mathrm{SE}_{s}\left\{ \overline{\dt[\hat\theta]{{Y_H}}{ R}}_{Z=1} -
                             \overline{\dt[\hat\theta]{{Y_H}}{ R}}_{Z=0}
                           \right\} }
\end{equation}
to a $t$ or standard Normal reference distribution.
Sandwich standard errors confer robustness to misspecification of
$\var(\dt[\hat\theta]{Y_{H}}{R}\mid  R)$, not of $\EE(Y_{H}| R)$ \citep{freedman2006sch}, the latter being the topic of
Section~\ref{sec:robust-altern-ordin}.

In applying \textsc{ancova} to an RCT, $\mathrm{SE}_{s}$ in
\eqref{eq:tedef} may be taken as \eqref{eq:sudef}'s
$\mathrm{SE}_{u}$% \citep{samii2012equivalencies}
, a more straightforward quantity to compute; unfortunately, in RDDs
this is not the case
(Section~\ref{apnd:requ-forpr-eqref}).  On the
other hand, in \textsc{ancova} as applied to RDDs, if standard
errors are calculated by the sandwich method then under
$H: Y_{T} = Y_{C}$, \eqref{eq:tedef} is algebraically equivalent to the
$t$-statistic of contrast $\hat{\alpha}_{1} -\hat{\alpha}_{0}$.


% \subsection{Restricting the Window of Analysis in
%   RDDs}\label{sec:window}


% %% For instance, \citet{thistlethwaite1960regression} studied the effect
% %% of receiving a National Merit Scholarship on students' probabilities
% %% of attending graduate school.
% %% Presumably, no one is considering awarding merit scholarships to students with particularly
% %% low PSAT scores.
% %% For these students, the hypothetical effect of merit scholarships is
% %% not relevant.
% %% The graduate school choices of students who who barely qualified, or
% %% barely missed qualifying, though, could be quite interesting.
% %% In these cases, both statistical and substantive considerations
% %% recommend restricting analysis to a window $\mathcal{W}$ around the
% %% cutoff.

% Adaptive choice of $\mathcal{W}$ is an open topic of research.
% \citet{imbens2012optimal} produce regularized, non-parametric
% estimates of the curvature of $\EE(Y | R=r)$ in the vicinity of the
% cutpoint, using these to identify bandwidths approximately minimizing
% the mean-squared error of
% $\lim_{r\downarrow c} \EE(Y | R=r) - \lim_{r\uparrow c} \EE(Y |
% R=r)$. %referenced in r1 Reply to AE
% An alternative approach relies on specification checks, refining
% $\mathcal{W}$ until it passes a designated test or battery of tests.
% The initial choice of $\mathcal{W}$ may emerge from substantive
% motivation; other analysts start with a $\mathcal{W}$ that's likely to
% be too wide, expecting the specification tests to force them to narrow
% it.  For example, let $H_b$ denote the hypothesis that the analysis
% model assumptions hold for $\{i \in \mathcal{S}:R_i\in
% \mathcal{W}_b\}$, where $\mathcal{W}_{b} =  (c-b,c+b)$.
% \citet{cattaneo2014randomization} recommend testing $H_{b}$'s implications for covariate
% balance at a sequence of bandwidths $b$, starting with a very large
% $b$, $b_{\mathrm{max}}$, and ending with the largest $b$ such that $H_{b}$
% is not rejected, $b^{*}$; their recommended window of analysis is
% $\mathcal{W}_{b^*}$.
% \citet{liMatteiMealli2015BayesianRD} recommend a similar approach
% based on Bayesian posterior probabilities of covariate balance for
% subjects within a sequence of possible bandwidths.

% %% Concerns about multiple testing sometimes arise, since any number of
% %% bandwidths may have been tested en route to determining the largest
% %% $b$ for which $H_{b}$ is sustained; but it so happens that the
% %% procedure determining the largest non-rejected bandwidth implicitly
% %% corrects for multiplicity.  Let $\mathcal{B} \subseteq (0, \infty)$ represent a finite set of
% %% candidate bandwidths, with an accompanying family $\mathcal{F}$ of
% %% tests, one for each $H_{b}$. Define a modified family $\mathcal{F}^{*}$
% %% as follows: for each $b \in \mathcal{B}$, $H_{b}$ is rejected at level
% %% $\alpha$ if only if at level $\alpha$, $\mathcal{F}$ rejects each of
% %% $\{ H_{b'}: b'\in\mathcal{B}, b'\geq b\}$. The
% %% testing in order principle \citep[Proposition~1]{rosenbaum2008testing}
% %% states that if each test in $\mathcal{F}$ tests its corresponding
% %% $H_{b}$ with size $\alpha$ then, with probability $1-\alpha$ or more,
% %% $\mathcal{F}^{*}$ rejects only hypotheses $\{H_{b}: b \in \mathcal{B}\}$
% %% that happen to be false --- $\mathcal{F}^*$ strongly controls the family-wise error rate at level $\alpha$.

% Given the variety of methods available for bandwidth selection, many
% methodologists recommend estimating treatment effects with each of a
% range of plausible bandwidths
% \citep[e.g.,][p.272]{eggers2015validity}.  It's natural to consider a
% bandwidth plausible only if covariates are well balanced within the
% corresponding window.
% %% When $b^{*}$ is the largest
% %% non-rejected bandwidth,
% %% $\mathcal{B}^{*} = \mathcal{B} \cap [0, b^{*}]$ suggests itself as
% %% such a range; each $H_{b}, b\in \mathcal{B}^{*}$ will have been
% %% sustained in a testing procedure with family-wise type 1 error of
% %% $\alpha$ or less.
% %% An analyst might combine tests $\mathcal{F}^*$ with still others,
% %% $\mathcal{G}$ say, to further narrow the plausible range
% %% $\mathcal{B}^*$; if tests in $\mathcal{F}$ maintained size
% %% $\alpha_\mathcal{F}$, while $\mathcal{G}$ strongly controls the
% %% family-wise error with size $\alpha_\mathcal{G}$, the overall size of
% %% the combined procedure is no greater than $\alpha_{\mathcal{F}} +
% %% \alpha_{\mathcal{G}}$.
% Such goodness of fit testing is often done with
% $\alpha$ thresholds other than $.05$; for instance, \citet{cattaneo2014randomization} use
% $\alpha=.15$.


\subsection{Robust Estimation of $\EE (Y_{C}| R)$}
\label{sec:robust-altern-ordin}
% Let the running variable be so
% centered that the threshold lies at $r=0$, as in their application.
% Under the variant of \textsc{ancova} relaxing
% $Y_{C} = \alpha_{0} + \beta R + \epsilon$ to
% $Y_{C} = \alpha_{0} + \beta_{-} \indicator{R<0} R + \beta_{+}
% \indicator{R\geq 0}R  + \epsilon$,
% the $Z$-coefficient in an OLS regression of $Y$
% on the interaction of $Z$ and $R$ remains unbiased for
% $\lim_{r\downarrow 0} \EE(Y | R=r) - \lim_{r\uparrow 0} \EE(Y |
% R=r)$, as well as $\EE (Y_{T} - Y_{C} )$.  (It also coincides with $\overline{\dt[\hat\theta]{{Y}}{
%   {R}}}_{Z=1} - \overline{\dt[\hat\theta]{{Y}}{ {R}}}_{Z=0}$, where
% $\dt[(a, b_{0}, b_{1})]{ \mathbf{y}}{ \mathrm{r}} = \mathbf{y} - a - (b_{0}
% \indicator{r < 0} + b_{1} \indicator{r\geq 0})\mathrm{r}$
% and $\hat\theta = (\hat\alpha, \hat{\beta}_{-}, \hat{\beta}_{+})$ are
% as estimated under the interaction model.)

% \citet{thistlethwaite1960regression} warned that \textsc{ancova}'s
% linearity assumptions would occasionally be too strong.
\sloppy
Assumptions about the form of $\EE[(Y_{T}, Y_{C}) | R]$ are not
readily dispensed with;
% Statistic (\ref{eq:tedef}) is often computed using a portion of the
% available data, a subset $\mathcal{W} \ni c$ of the support of $R$
% \citep[e.g.][]{imbens2008regression}.
even nonparametric RDD models
%take $R$ to follow a continuous distribution, and
place continuity and smoothness restrictions on $\EE(Y_{C}| R=r)$ \citep[\S~5]{lee2008regression, kolesarRothe17}.
On the other hand, methods discussed in Sec~\ref{sec:robust-analys-covar} continue to
apply if $\EE (Y_{C}| R) = \alpha + R\beta$ is relaxed to
$\EE (Y_{C}| R) = \alpha + \vec{f}(R)\beta$, for
$\vec{f}(\cdot)$ a $1 \times k$ vector valued function.
Unfortunately, if the model is fit by OLS, then such relaxation of assumptions can have the unwelcome
side-effect of undercutting the robustness of the analysis.  The
reasons have to do with mechanics of regression fitting.

Polynomial specifications
$\EE(Y | R=r) = \sum_{j=0}^{J} r^{j} \beta_{j}$ are common but often
problematic; in combination with ordinary least squares fitting, they
implicitly assign case weights that can vary widely and
counterintuitively \citep{gelman2016high}.
This liability is already
in evidence with $J=1$, the linear specification, where leverage
increases with the square of $r -\bar{r}$.  With an analysis sample
of the form $\{i : R_{i} \in \mathcal{W}\}$ for a ``window''
$\mathcal{W} \ni c$, if $\mathcal{W}$ is slightly too wide then the
sample is contaminated near its outer boundaries, precisely
where leverage is at its highest.

In order to identify leverage points that are also influential,
OLS fitting is sometimes combined with specialized diagnostics such as
plots of Cook's distances
 \citeyearpar{cook1982residuals}. An alternate remedy, playing
 an important role in this paper, is to fit the
 specification using a modern MM,  SM, or similar estimator so designed
 as to possess a bounded influence function
 \citep{yohaiZamar1997locallyrobustMestimates}; such procedures address influence
 in the course of the fitting process.
 % Diagnostics such as inspection of Cook's distances
 % \citeyearpar{cook1982residuals} can identify leverage points that are
 % also influential.  In regression problem other than RDDs, a natural
 % remedy is to select $\vec{f}(\cdot)$ as a spline basis, thus fitting
 % a piecewise rather than a global polynomial
 % \citep{ruppert2003semiparametric}.  For the purpose of estimating
 % $Z$'s contribution to $\EE (Y|R,Z)$, this is similar in its effects
 % to retaining the global polynomial specification while narrowing $\mathcal{W}$
 % until it does not straddle knots of the chosen spline basis.
In MM-estimation as in OLS,
coefficients $\beta$ of a linear specification solve estimating equations
$\sum_{i} \psi\left\{ ({y}_{i} -
\vec{x}_{i}\beta)/s \right\} \vec{x}_{i} =0$, where $s>0$ and
$\psi(\cdot)$ is an odd function satisfying $\psi(0)=0$,
$\psi'(0)=1$ and $t\psi(t)\geq 0$; bounded influence fitters replace OLS's $s\equiv 1$ with resistant preliminary
estimates of residual scale, and OLS's $\psi(t) = t$ with a continuous $\psi$
that satisfies $\int_{0}^{\infty}\psi(t)dt < \infty$. This limits
the loss incurred by the fitter for failing to adapt itself to a small
portion of aberrant observations;
% e.g. Huber's $t \mapsto \mathrm{sign}(t) * \min(|t|, 1.345)$
% \citeyearpar{huber1964robust}.  Capping the influence of any one observation
% requires that $\psi$ be not only bounded but
% \textit{redescending}. A redescending $\psi(t)$
% increases smoothly from 0 as $t$ increases from 0, but for
% sufficiently large $t$ it levels out and redescends, reaching 0 at
% some $t_{r} < \infty$; $\psi(t) = 0$ for all $t>t_{r}$.
it is permitted to instead systematically down-weight them. When
present, such aberrancies present large residuals, that are easier to
see in ordinary diagnostic plots, and small
``robustness weights,'' an additional diagnostic not available for OLS fitting
\citep{maronna2006robust}.  We are not aware of prior work addressing
potential contamination of an RDD sample with the assistance of
bounded influence MM-estimation.

Surprisingly, given their common origins in %the work of
Huber \citeyearpar{huber1964robust}, MM estimation is not routinely
paired with sandwich estimates of variance, as in \eqref{eq:tedef} and
the surrounding discussion of \S~\ref{sec:robust-analys-covar}.
Exceptions include Stata's \texttt{mmregress} and R's \texttt{lmrob},
which optionally provide Huber-White standard errors
\citep{verardiCroux2009robust,rousseuwetal2015robustbase}.

\subsection{Specification Tests}\label{sec:specification}

Analysis of RCTs and quasiexperiments often hinges on assumptions of
independence of
 $\mathbf{Z}$ from $(\mathbf{X}, \mathbf{Y}_{C}, \mathbf{Y}_{T})$.
%% Analogous statements will hold for more complex randomization or
%% ignorability assumptions as well.
 Although neither $\mathbf{Z} \independent \mathbf{Y}_{C}$ nor
 $\mathbf{Z} \independent \mathbf{Y}_{T}$ can be directly tested,
 since potential outcomes are only partly observed, assumptions of form
 $\mathbf{Z} \independent \mathbf{X}$ are falsifiable: researchers can
 conduct placebo tests for effects of $Z$ on $X$.
% Just as $H: \mathbf{y}_{T} - \mathbf{y}_{C} =
% \mathbf{\tau}$ is tested in the strong randomization inference
% paradigm, the researcher might assess imbalance in a covariate, as measured
% either by the difference of means,
% $\bar{x}_{Z=1} - \bar{x}_{Z=0}$, or by the difference of means in a transform of
% the covariate, $\overline{\dt[\thetaInf]{\mathbf{x}}{\mathbf{r}}}_{Z=1} - \overline{\dt[\thetaInf]{\mathbf{x}}{\mathbf{r}}}_{Z=0}$,
% to the distribution of the same difference under permutations of
% $\mathbf{z} $.
Of course, treatment cannot affect pre-treatment variables; this is
model-checking (%\citealp{bayarriBerger2000pvalues};
\citealp[][\S~5.13]{cox2006pos}%; \citealp[][Ch.~6]{gelman:etal:2004};
%\citealp[][Ch.~14]{lehmannRomano2006TSH}
).  Here as
 elsewhere \citep{box1980sab}, checks finding fault with a model prompt
 refinement, not abandonment; refinements typical of RDD analysis remove portions of
 the sample that are suspected of contamination.

%% BELOW APPEARED IN R2
% With $k>1$ covariates $\mathbf{x}_{1}, \ldots, \mathbf{x}_{k}$, issues
% of multiplicity come into play: if a test of the hypothesis that $(X_{1}, \ldots, X_{k})$
% are independent of $Z$ rejects whenever any of the $k$
% level $\alpha$ tests is rejected, then its type 1 error rate may
% greatly exceed $\alpha$.  Because the tests are non-independent, with
% covariances that (except in the differences-of-means case) may be
% difficult to estimate, we combine them with a simple Bonferroni
% correction.
%% BELOW HAD BEEN WRITTEN FOR R2 AND THEN EDITED OUT
%% One solution is to combine the $p$
%% test statistics $\overline{(\mathbf{x}_{1})}_{z=1} - \overline{(\mathbf{x}_{1})}_{z=0}$, \ldots, $\overline{(\mathbf{x}_{p})}_{z=1} - \overline{(\mathbf{x}_{p})}_{z=0}$
%% into one statistic $t(\mathbf{x}, \mathbf{z})$, then conduct a
%% single test based on it.
%% %For comparing
%% %3 or more treatment conditions while controlling the family-wise error rate, a
%% %common device is to use the ``max-t'' statistic, i.e. test based on $t(\mathbf{x},
%% %\mathbf{z}) = \max_{k} (t(\mathbf{x}_{\cdot,k}) -
%% %\EE_{0}T(\mathbf{x}_{\cdot,k}))/\var^{1/2}_{0}(T(\mathbf{x}_{\cdot,k}))
%% %$ \citep{hothorn2008simultaneous};
%% %In a discussion of covariate balance specifically,
%% \citet{hansen:bowers:2008} recommend a combined statistic
%% along the lines of Hotelling's
%% \citeyearpar{hotelling1931generalization} $T^2$, $d^{2}(\mathbf{x}, \mathbf{Z})$, that follows a large-sample chi-squared
%% distribution under the hypothesis that $\mathbf{X} $ is
%% independent of, and unaffected by, $\mathbf{Z}$.
%% %% Other solutions
%% \citep[\textit{e.g.,}][]{rosenbaum2005exact} involve testing with a
%% statistic $t(\mathbf{x}, \mathbf{z})$ that is not directly related to
%% the univariate statistics.

 Writing in the RDD context, \citet{cattaneo2014randomization} test
 for marginal associations of $\mathbf{Z}$ with covariates $\mathbf{X}_{i}$,
 $i=1, \ldots, k$, using the permutational methods that are applied
 in Fisherian analysis of RCTs.
% Specifically, they
%  obtain p-values though \eqref{eq:01}, with test statistics
%  $t(\mathbf{x}_{i}, \mathbf{z}) = |\overline{(x_{i})}_{z=1} -
%  \overline{(x_{i})}_{z=0}|$,
%  $ i \leq k$, after conditioning on $\mathbf{X}$ in addition to
%  $\sum_{i} Z_{i}$.
Relatedly, \citet{lee2010regression} recommend a
 tests for conditional association, given $R$, of $\mathbf{Z}$ and
 $\mathbf{X}$, by fitting models like those discussed in
 \S~\ref{sec:robust-analys-covar} for impact estimation, but with
 covariates rather than outcomes as independent variables%
%, they incorporate covariance adjustments for $R$
. %, testing
 %whether these regressions' $z$ coefficients are plausibly
 %0.
Viewing the $R$-slopes and intercepts as simultaneously estimated
 nuisance parameters, these are balance tests applied to
 the covariates' residuals, %net of linear association with $R$,
rather than to the covariates themselves.

%Whether covariates or covariate residuals are assessed for lack of
%correlation with $Z$,
If there are multiple
covariates there will be several such tests. To summarize their
findings with a single p-value, the
regressions themselves may be fused within a
``seemingly unrelated regressions'' model, followed by a test of the
intersection hypothesis that each regression's $z$-coefficient is 0
\citep{lee2010regression}, or the
separate tests' p-values could simply be combined using the Bonferroni
principle.  Li et al.'s
\citeyearpar{liMatteiMealli2015BayesianRD} RDD method wraps
its specification checking steps into a Bayesian model. % that is also used for effect
%estimation.
This approach addresses multiplicity of placebo tests as a matter of course; as with the method of \citet{cattaneo2014randomization}, however, its implied placebo tests, of whether $Z$ ``affects'' $\mathbf{X}$, are not covariance adjusted for $R$.


McCrary's test for manipulation of treatment assignments
\citeyearpar{mccrary2008manipulation} can be understood as a %lack of
%association test applied over an $\{|R| < b\}$ window.
placebo test with the density of $R$ as the independent variable.
The test's
purpose is to expose the circumstance of subjects finely manipulating their
$R$ values in order to secure or avoid assignment to treatment.  Absent
such a circumstance, if $R$ has a density then it should appear to be
roughly the same just below and above the cutpoint.  McCrary's
\citeyearpar{mccrary2008manipulation} test statistic is the difference
in logs of two estimates of $R$'s density at 0, based on observations
with $R<0$ and $R>0$ respectively.
% Inspection of the form of these
% estimates reveals that, but for the presence of the logarithms, the
% statistic also is a difference of means over observations with
% $Z_{i}=0$ or $1$, in both cases with $|R|$ as the quantity being
% averaged.  (The two means are calculated with weighting, with weights
% %set to 0 for $|R|>b$ and otherwise
% determined by the combination of a
% chosen kernel and bandwidth and simultaneously estimated nuisance parameters; see
% McCrary's [\citeyear{mccrary2008manipulation}] equation 4.)
Manipulation is expected to generate a clump just beside the cut
point, on one side of it but not the other, and this in turn engenders
imbalance in terms of distance from the cut-point.


In practice, specification test failures inform sample exclusions.
%: failed density tests lead to exclusions around the cut
%point; failed placebo tests lead to narrower widths of analytic
%windows.
Failures of the density test are addressed by restricting
estimation to observations with $|R|>a$, some $a \geq 0$
\citep[e.g.,][]{barrecaetal2011birthweightRDD,eggers2014validity}.
When balance tests fail,
\citet{lee2010regression} would select a window
$\mathcal{W} \subseteq [-b, b]$, $b>0$, % using
% a method that anticipates possible non-linearity in $\EE (X | R=r)$
% outside a neighborhood of 0,
and repeat the test on $\{i : r_{i} \in \mathcal{W}\}$. Similarly,
\citet{cattaneo2014randomization} recommend testing hypotheses
$H_{b_{i}}: {X} \independent {Z}| \{|R| < b_{i}\}$, $b_1 > b_2 > \cdots >0$, in sequence,
proceeding until identifying  $b = \max \{ b_i: H_{b_{i}}$ is not rejected$\}$;
the analytic sample is then restricted to  $\{i: r_{i} \in (-b,b)\}$.
%This method's tests of each $H_{b_{i}}$ are neither covariance
%adjusted nor multiplicity corrected.
The analyst decides whether this test-and-reduce process
should begin in the middle of $\mathcal{W}$ or at its boundaries.

% The $\alpha$ level used for these tests may differ from the
% $.05$ that is conventional in outcome analysis;  \citet{cattaneo2014randomization} use $\alpha=.15$.
% Whatever the choice of $\alpha$, Berger's sequential intersection
% union principle \citep{rosenbaum2008testing,hansenSales2015cochran} entails that the sizes of $H_{b*}$, and of each
% of the $H_{b}$ rejected prior to testing $H_{b^{*}}$, are not inflated
% by the multiplicity of tests within the family $\{H_{b}: b\}$.

% McCrary's specification test focuses on the center of the RDD window,
% rather than its boundaries \citep{mccrary2008manipulation}.
% It is designed to test whether subjects consciously sort around the
% cutoff by manipulating their $R$ values explicitly.
% It does so by examining whether an unexpectedly large or small number
% of subjects find themselves just barely on one side of the cutoff or
% the other.
% When the running variable is discrete, for each value $r$ in the support of $R$,
% let $n_R(r)$ denote the number of subjects $i$
% with $R_i=r$, i.e.,  $|\{i:r_i=r\}|$; continuous $R$s are binned prior to the
% test.  Next, $n_{R}$ is used as the outcome of a preliminary RDD
% analysis, modeled using local linear regression as a function
% of $r$ on either side of $c$. The test asks whether there is a discontinuity at $c$:
% a change in $n_{R}$ larger than chance could produce suggests subjects
% may have manipulated $R$ to control their treatment assignments,
% perhaps invalidating the analogy to a controlled experiment.
%% [clipping the below to save a little space; I think it works w/o it.]
%% A McCrary test failure is neither necessary nor sufficient to show
%% that subjects sorted around the cutoff; however, it is suspicious.

\section{Randomness and Regression in RDDs}\label{sec:theMethod}

%% RDDs differ from RCTs in an important way:
%% $Y_C$ typically correlates with $R$, and is therefore not independent
%% of $Z$; Strong Ignorability \eqref{eq:ignore} is violated.
%% Can it be relaxed to an assumption that is
%% simultaneously plausible in RDDs and compatible with similar methods?

When circumstances call for generating random numbers, such as in
RCTs or Monte Carlo methods, statisticians often settle instead for
pseudo-random numbers, deterministic
outputs of a complex algorithm and an initial seed; a person in
possession of that seed can exactly predict the pseudo-random draws.
``True'' random numbers, on the other hand, derive from physical
processes that are either chaotic \citep[e.g.][]{uchida2008fast} or
quantum \citep[e.g.][]{stefanov2000optical}, and are not
predictable.  However, these processes often contain an element
of predictability; % \citep{raz2005extractors};
true random
numbers are obtained only after separating predictable from
non-predictable components \citep[see, e.g.][]{Nisan1999148}.%,vadhan2012pseudorandomness}.

Similarly, RDD analysis decomposes $Y_{T}$ and $Y_C$ as the sum of systematic
components, a function of $R$, and disturbances, relying on the nature of the decomposition to ensure
that, vis a vis one another, the disturbances and $Z$ may
be regarded as random.
% There is no need
% to assume the systematic component to equal or be consistent for
% $\EE (Y_{C}| R)$, nor to make further distributional assumptions on
% the disturbances.

\subsection{An Analytic Model for RDDs} \label{sec:model-eey-c-r}

Suppose the statistician to have selected a \textit{detrending procedure}: a
trend fitter, i.e. a function of
$\{({y}_{i},d_{i},r_{i})\}_{i=1}^{n}$ returning
fitted parameters $\hat{\theta}$ in a sufficiently regular
fashion, along with a
family $\{\dt{\cdot}{\cdot}: \theta\}$ of residual or partial
residual transformations, each mapping data $(\mathbf{y}, \mathbf{r})$ to
$\{\dt {y_{i}}{\mathbf{r}} \}_{i=1}^{n}$.
Appendix~\ref{sec:large-sample-rand}
%\citet[\S~\ref{sec:large-sample-rand}]{lrdauthors:supp}
states the
needed regularity condition, which is ordinarily met by OLS and always
met with our preferred fitters (Section~\ref{sec:test-hypoth-no}).
Causal inferences taking
the perspective that $Z$ is random due to
randomness in $R$ are then possible under the following assumption.

\begin{ass}{Residual Ignorability}
\sloppy
Given $\mathcal{W}$ with $0< \PP( R \in
\mathcal{W}, R\leq 0) < 1$ and a detrending procedure $(\hat{\theta}, \dt{y}{r})$,
\begin{equation}\label{ycheck}
\dt[\thetaInf]{Y_{C}}{ R }%\mathbf{\Ych}_{C}
\independent {Z}| \{R \in \mathcal{W}\},
\end{equation}
where $\thetaInf$ is the probability limit of $\hat\theta$.
\end{ass}
Residual Ignorability states that, though $Y_C$ may not be independent of
$Z$,  it admits a residual transformation bringing about such
independence.   With $\dt[\hat\theta]{Y_{C}}{ R}$ a suitable
partial residual, Residual Ignorability is entailed by the
\textsc{ancova} model (\S~\ref{sec:robust-analys-covar}), or by the combination of any parametric model
for $\EE (Y_{C}| R)$ with a strict null $H$ relative to which the
value of $Y_{C}$ can be reconstructed from the values of $Y$, $D$ and
$Z$ (\S~\ref{sec:randProc}).
% or where $H$ asserts $Y_{T} = Y_{C} + \tau D$ for the true CACE  $\tau
% = \EE(Y_{T}-Y_{C}| D_{T}=1)$,  and in addition
% $(Y_{T} - Y_{C}) \independent R | D_{T}$ holds.
(In either of these cases $\dt[\thetaInf]{Y_{C}}{ R}$
is independent not only of $Z$ but also $R$,
a modest strengthening of~\eqref{ycheck}.)
% relaxes
% this condition, accommodating cases in which
% $Y_{C} - \EE (Y_{C}|R)$ equals $\dt[\hat\theta]{Y_{C}}{ R}$ only
% in the on-average sense of conditional expectation given $Z$.
% Just as an accurate
% specification of %$\EE (Y | Z, X)$ or
% $\EE (Y_{C}| X)$ is helpful but
% not necessary for valid covariate adjustment in an RCT,  it is helpful
% but not necessary for Residual
% Ignorability that $\dt[\thetaInf]{Y_{C}}{R}$ be a function of
% $Y_{C}$'s residual from a consistent estimate of $\EE (Y_{C}| R)$.
% If it is, then Residual Ignorability holds with $\dt[\thetaInf}]{y}{r} = y-
% \EE(Y_{C}| R=r)$ and $\hat\theta$ a standard moment- or
% likelihood-based estimate of the parameters of $Y_{C}$'s regression on
% $R$.  If not, however, one may
% still have $\dt[\thetaInf]{Y_{C}}{ R}$ independent of $Z$, or even
% $\dt[\thetaInf]{Y_{C}}{ R} $ independent of $R$
% with some
% other choice of $e_{\theta}(\cdot)$ and $\hat\theta$.   It is neither required nor preferred that model fit be appraised in terms of squared
% error loss. Taking $\theta = (s, \mathbf{\beta})$ and
% $\dt[\theta]{y}{r} = \psi\big\{ (y - f(r) \mathbf{\beta} )/{s}\big\} $,
% with $f(\cdot)$ scalar- or vector-valued and
% $\psi(\cdot)$ a scoring intended to convey robustness to contamination of the
% $\EE ( Y_{C}| R)$ model, is quite consistent with \eqref{ycheck}.

% To assume \eqref{ycheck} is in no way to assume
% %\begin{equation} \label{eq:notycheck}
% $\dt[\hat\theta]{Y_{C}}{ R} \independent {Z}| \{R \in \mathcal{W}\}$.
% %\end{equation}
% The simplification of \eqref{ycheck} replacing its
% $\thetaInf$ with $\hat\theta$ cannot be expected to hold: unless
% $\hat\theta -\thetaInf$ has zero variance, it is almost inevitable
% that it correlate in some way with $Z$, and by extension that $\dt[\hat\theta]{Y_{C}}{ R}$ and
% $Z$ be non-independent.
% Rather, Residual Ignorability presumes
% existence, if not knowledge, of a single $\theta = \thetaInf$ relative
% to which potential partial residuals
% %$\dt[\thetaInf]{Y_{T} - \tau D_{T}}{ R}$ and
% $\dt[\thetaInf]{{Y}_{C}}{ R}$ and treatment assignment $Z$ are
% mutually independent.
% In marked contrast with RCTs, in typical RDDs there can be at
% most one $\theta$ for which this is true. To see this, consider
% $\dt[\theta]{y}{r} = y - \theta r$.  If $Z$ were jointly independent
% of $\dt[\theta_{0}]{Y_{C}}{ R}$ and $\dt[\theta_{1}]{Y_{C}}{ R}$, it would also have
% to be independent of $(\theta_{0} - \theta_{1}) R$. If $\theta_{0}
% \neq \theta_{1}$ this would entail $Z \independent R$ --- a
% contradiction, since $Z = \indicator{R<0}$.  Were $R$ a baseline variable
% in an RCT, $\dt[\theta]{Y_{C}}{ R} \independent Z$ would hold for all $\theta$,
% and inferences relying on this independence would require neither covariance adjustment
% nor consistent estimation of covariance parameters --- both of which are essential for RDDs.

\sloppy
Assuming Residual Ignorability, inference about treatment effects is
made conditionally, on
$\mathbf{A}= (\dt[\thetaInf]{\mathbf{Y}_{C}}{ \mathbf{R}}$, $\mathbf{D}_{T},
\{(Y_{Ti}, Y_{Ci}, D_{Ti}, R_{i}) \indicator{{R}_{i} \not\in
  \mathcal{W}}\}_{i=1}^{n})$.
Conditioning on the full data vector when $R \not\in \mathcal{W}$
excludes observations for which \eqref{ycheck} is not assumed.
Conditioning on
$\dt[\thetaInf]{\mathbf{Y}_{C}}{ \mathbf{R}}$
removes little of the entropy
in $\mathbf{R}$, leaving it available as a basis for inference; in
contrast, conditioning on Section~\ref{sec:randProc}'s $(N_{0}, N_{1}, \mathbf{Y}_{C})$ would
severely constrain $\mathbf{R}$.
Uncoupled to $Y_{T}$s, the detrended  $Y_{C}$s,
$\dt[\thetaInf]{\mathbf{Y}_{C}}{ \mathbf{R}}$,
are in themselves uninformative about $\EE(Y_{T} - Y_{C})$, so
the variables comprising $\mathbf{A}$ are jointly
%\marginpar{$\leftarrow \Delta$ to $(\mathbf{Y}_{C}, N_{0}, N_{1})$?}
S-ancillary, just as $\mathbf{A}^{\dagger}$ was seen to be
in Section~\ref{sec:randProc}.  As in Neyman-style randomization inference for RCTs, some conditioning variables are
unobserved; but this is not an impediment, at least for large-sample
inferences.


% Because \eqref{ycheck} can be true for at most one
% $\thetaInf$, in applications it must be paired with an assumption that
% one has a estimation routine that consistently estimates $\thetaInf$.  As
% discussed in Sec.~\ref{sec:robust-analys-covar}, in RDDs the standard error accompanying
% $\overline{\dt[\hat{\theta}]{\bty }{ \mathbf{r}}}_{z=1} -
% \overline{\dt[\hat{\theta}]{\bty }{ \mathbf{r}}}_{z=0}$ must
% reflect the indirect contribution to its sampling variability of errors of estimation of
% $\thetaInf$, as the $\mathrm{SE}_{s}$ of \eqref{eq:tedef}
% does but the $\mathrm{SE}_{u}$ of \eqref{eq:tudef} does not.

\begin{comment}
To see this, consider
$\dt[{\theta}]{y }{ r} = y - r \theta$.  Up to an $o_{P}(n^{-1/2})$
error,
\begin{equation}\label{eq:2}
    \{ \overline{\dt[\hat{\theta}]{\bty }{ \mathbf{r}}}_{z=1} -
\overline{\dt[\hat{\theta}]{\bty }{ \mathbf{r}}}_{z=0} \} -
\{  \overline{\dt[\thetaInf ]{\bty }{ \mathbf{r}}}_{z=1} -
\overline{\dt[\thetaInf]{\bty }{ \mathbf{r}}}_{z=0} \}   \approx
\Big[ \EE \big\{ R \big| Z=1 \big\} - \EE
  \big\{R \big| Z=0 \big\} \Big]  (\hat\theta -
\thetaInf)^{t}.
\end{equation}
In an RCT the term in square brackets at right would be 0, ensuring that even if
variation in $\hat\theta - \thetaInf$ is positive, its large-sample
contribution to $\overline{\dt[\hat{\theta}]{\bty }{ \mathbf{r}}}_{z=1}
- \overline{\dt[\hat{\theta}]{\bty }{ \mathbf{r}}}_{z=0}$  is
$o_{P}(n^{-1/2})$, and is negligible; but in an RDD it is
intrinsically nonzero, and \eqref{eq:2} is $O_{P}(n^{-1/2})$.
 For the same reason, permutation
tests do not provide exact inference for RDDs: in an RDD, even under
$H$ conditioning on %$\mathbf{Y}_{C}$'s and  $\mathbf{Z}$'s order statistics
$(\mathbf{Y}_{C}, N_{0}, N_{1})$
does not eliminate variability in $\hat\theta$; the differences
at left of \eqref{eq:2} differ from one another. (In Randles's
[\citeyear{randles:1982}] %analysis of errors due to estimation of secondary parameters,
terms, RCTs give rise to ``Case A'' distributions whereas RDDs' are ``Case B.'' Only in the unusual circumstance that $\thetaInf$ is externally
determined, and is known rather than estimated, can \eqref{ycheck}
serve as a basis for exact tests.)  Asymptotically
distribution-free inferences, on the other hand, are straightforwardly
arranged by estimating
$\overline{\dt[\hat{\theta}]{\bty }{ \mathbf{r}}}_{Z=1} -
\overline{\dt[\hat{\theta}]{\bty }{ \mathbf{r}}}_{Z=0}$'s standard error
with attention to sampling variability in $\hat\theta$,  as
$\mathrm{SE}_{s}$ does.
\end{comment}

% If available, a lagged, pre-treatment version of the outcome that's
% distinct from the running variable may be the basis for particularly
% informative plots and regression diagnostics
% \citep{maynard2013strengthening}.

% Such M-estimation conveys robustness to other errors of specification
% that are quite relevant to RDDs.  In modern robust M-estimation
% \citep{maronna2006robust}, estimating equations defining the
% regression coefficients are adjusted so as to bound the influence of
% any one or small number of observations.  Under uncertainty about
% width of the analysis window $\mathcal{W}$, it will be safer to fit
% $\hat\theta$ using such a method, the points of greatest leverage
% generally being the closest to the edge of $\mathcal{W}$.  Thus
% robust regression addresses a fundamental incompatibility between RDDs
% and ordinary least squares, namely that the observations whose
% suitability for inclusion in the analytic sample is most questionable
% --- those whose values of the running variable are farthest from the
% cutpoint --- also exert the greatest leverage in estimation.



\subsection{Checking and Refining the Window ($\mathcal{W}$)}
\label{sec:bandwidthChoice}

If subject matter knowledge suggests that chance variability intrinsic
to $R$ is typically of magnitude $b$, then RDD modeling focused on
distilling local randomization might set $\mathcal{W}$ to $[-b,b]$. But it
is also sensible to subject such an initial choice to specification
testing (\S~\ref{sec:specification}).%\marginpar{I replaced $c$ here
%  with $b$ since $c$ is the cutoff-AS\\ ---Good catch thnx! -B}

Common RDD specification checks
can be regarded as testing Residual Ignorability with a multivariate
``outcome'' $Y^{*}$ combining the actual outcome $Y$ with covariates $X$---\eqref{ycheck} with
$\mathbf{Y}_{C}^{*} = (\mathbf{X}, {Y}_C)$ in place of $Y_{C}$.
% perhaps with minor adjustment to the forms of $\dt[\theta]{ \cdot }{ r}$
% and corresponding fitter $\hat{\theta}(\cdot, R)$.
%
%In addition to $\mathcal{W}$, researchers must pick $f$, a model for $Y_C$.
%% If a lagged outcome is not available, the data will not have informed the specification of
%% $\EE (Y_{C}| R=r) $ on the treatment side of the threshold.  It
%% should be treated as an approximation, and $\mathcal{W}$
%% deliberately chosen to be narrow enough to limit influences of errors
%% of approximation, as distinct from errors of estimation, on statistics
%% $t({\mathbf{y}_H} - f(\mathbf{r}; \theta(\mathrm{r},
%% {\mathbf{y}_H}))$.
%% Covariate placebo tests can be helpful in this regard.
This calls for preliminary detrending procedures, mechanisms to
decompose  $X$ into components that are systematic or unpredictable,
vis a vis $\mathbf{R}$, just as ${\mathbf{Y}_C}$ will later be decomposed.
%% Since $Z$ cannot have an effect on a pre-treatment
%% covariate, the preliminary step of reconstructing
%% $x_{C}$ is not needed.
%%% [Reorg obviated need for the above, as this passage now precedes
%%% discussion of reconstructing Y_C]
Our analysis of the LSO data posits systematic components that are
linear and logistic-linear in $R$, depending on whether $X$ is
a measurement or binary variable. %, paralleling its choice of a linear decomposition
% for the primary outcome, subsequent GPA.  (It is fitting that the $X$- and
% $Y$-specifications mirror each other in model complexity.)
The placebo check adds $Z$ to the specification and tests whether its
coefficient is zero.  We implement these checks as Wald tests with
heteroskedasticity-robust standard errors, as in
\S~\ref{sec:robust-analys-covar}, using the Bonferroni method to
combine placebo checks across covariates.
% yielding a test of the
% hypothesis that all of the covariates' residuals are jointly ignorable within the
% given $\mathcal{W}$.

If the first window tested has form $\mathcal{W} = (-b, b)$,
write $H_{b}$ for the corresponding joint ignorability hypothesis.
If the $R$-adjusted covariate placebo tests reject $H_{b}$ then the process is repeated for
$\mathcal{W}' = (-b', b')$, some
$b' < b$, and perhaps repeated again if $H_{b'}$ also is rejected.
This may seem to call for a further layer of multiplicity correction,
since any number of
bandwidths may have been tested before identifying a
$b$ for which $H_{b}$ is sustained; but it so happens that this form
of sequential testing implicitly corrects for multiplicity, according to the
sequential intersection union principle
(\citealp[SIUP;][Proposition~1]{rosenbaum2008testing};
\citealp{hansenSales2015cochran}). To compensate for conservatism of
the Bonferroni method, we test with size
$\alpha_{B}=.15$, not $.05$.
% Let
% $\{b_{1}, b_{2}, \ldots\} \subseteq (0, \infty)$ be a descending sequence of
% candidate bandwidths, with an accompanying family $\mathcal{B}$ of
% tests, one for each $H_{b}$. Define a modified family $\mathcal{B}^{*}$
% as follows: for each $b_{i}$, $H_{b_{i}}$ is rejected at level
% $\alpha$ if only if at level $\alpha$, $\mathcal{B}$ rejects each of
% $\{ H_{b_{j}}: j \leq i\}$. The
% sequential intersection union principle \citep[SIUP;][Proposition~1]{rosenbaum2008testing}
% states that if each test in $\mathcal{B}$ tests its corresponding
% $H_{b}$ with size $\alpha$ then, with probability $1-\alpha$ or more,
% $\mathcal{B}^{*}$ rejects only hypotheses $\{H_{b_{i}}:  i\}$
% that are false --- i.e., $\mathcal{B}^*$ strongly controls the
% family-wise error rate at level $\alpha$ \citep{hansenSales2015cochran}.

Writing $b$ for the half-width selected in this manner, i.e. $b=\max$
\{$b_{i}: H_{b_{i}}$ is not rejected\}, we next apply a McCrary
manipulation test to $(-b,b)$.  If this returns a $p$-value
$p_{0} < \alpha_{M}=.05$, we repeat it within windows
$\{ i: |R_{i}| \in (a_{j}, b)\}$, $0 = a_{0} < a_{1}< \cdots < b$,
terminating the process at the first $j$ for which
$p_{j} \geq \alpha_{M}$.  By a second application of the SIUP,
the size of this test sequence is $\alpha_{M}$.  Taken
together, placebo and McCrary tests restrict the sample to
$\mathcal{W} = (-b, b)$ or  $(-b, -a) \cup (a, b)$.
% Although subsequent
% diagnosis of outcome regression may in principle indicate suggest
% further exclusions, the analytic sample is expected to be $\{i: R_{i} \in \mathcal{W}\}$.

% Maximum likelihood estimation is one option for
% $\hat{\theta}_{x}(\cdot, \cdot)$, but
% %% \footnote%
% %% { In the special case that $\mathbf{x} $ is decomposed as the sum of a
% %%   linear function of $\mathrm{r}$ plus a residual
% %%   $\mathbf{x}^{\perp}$, with the linear function fitted by ordinary
% %%   least squares, $d^{2}(\mathbf{x}^{\perp}, \mathbf{z})$ can be
% %%   expediently calculated as the difference of
% %%   $d^{2}((\mathbf{x}, \mathrm{r}), \mathbf{z})$ and
% %%   $d^{2}(\mathrm{r}, \mathrm{z}) $.  }
% we prefer robust linear and logistic fitters
% \citep{rousseuwetal2015robustbase}, deeming their lesser contamination
% sensitivity a relevant advantage. Specifically, if the $\mathcal{W}$ under
% consideration is somewhat too wide, then a robustly fitted
% $\hat{\theta}$ still estimates the same $\thetaInf$ that would be
% estimated for $\mathcal{W}_{0} \subseteq \mathcal{W}$ narrow enough for covariate
% ignorability to hold. Similarly, to limit the possibility that the
% fitting routine would obscure differences between residuals above and
% beneath the threshold, we fit $R$-slopes using a specification allowing for an independent contribution from $Z$, but then set the $Z$-contribution to zero when decomposing the covariate.


\subsection{Inference About the Treatment Effect}
\label{sec:test-hypoth-no}

For inference about $\tau$ under the model
$Y_{T} = Y_{C} + \tau D_{T}$, select a specification
$\mu_{\beta}(\cdot)$ for $\EE(Y_{C}| R)$, %in the application of
%Section~\ref{sec:appl-effect-acad}
such as the
linear model $\mu_{\beta}(R) =\beta_{0} + R\beta_{1}$.
%Although a maximum likelihood
%estimate of $\beta$ may be available,
If moderate sample contamination may be present --- specifically, contamination of
a $O(n^{-1/2})$-sized
share of the sample --- consistent estimation of $\thetaInf$ requires a
bounded influence fitter (\citealp[Thm.~3]{he1991localbreakdown};
\citealp{yohaiZamar1997locallyrobustMestimates}), as opposed to
maximum likelihood.  OLS does not meet this consistency requirement;
nor do many robust regression methods engineered to meet
objectives other than bounding the influence function
\citep{stefanski1991note}.
Further, no specification test is powerful enough
to reliably detect contamination of this size; power to detect anomalies
affecting only $O(n^{-1/2})$ of the sample can only tend to a number
strictly less than 1.  %NB: This is expanded a little in a
                       %now-commented-out passage of Discussion, the
                       %on referencing vdvaart:1998.
Thus under uncertainty about the
proper limits of $\mathcal{W}$, MM-estimation (\S~\ref{sec:robust-altern-ordin}) is to be preferred.
The analyses and simulations presented below use MM-estimators with bisquare $\psi$ and
the ``fast S'' initialization of \citet{salibian-barreraYohai2006fastS}.

Separately for each hypothesis $H: \tau=\tau_0$ under
consideration, one calculates
$\mathbf{y}_{H} = \mathbf{y} - \mathbf{d}\tau_{0}$, then
applies the chosen specification and fitter to
$(\mathbf{y}_{H}, \mathbf{r})$.
The combination of the data, the
model fit, and the residual transformation $\dt{\cdot}{\cdot}$ give rise to residuals
$\dt[\hat\theta]{\mathbf{y}_{H}}{\mathbf{r}}$, completing the
detrending procedure. Whether $H$ is rejected or sustained is
determined by the value of the sandwich-based \textsc{ancova} $t$-statistic
\eqref{eq:tedef}.

In practice it is expedient to use a near-equivalent
test by modifying the detrending
procedure%, without need to separately calculate
%$\overline{\dt[\hat\theta]{{y_H}}{ r}}_{z=1} -
%\overline{\dt[\hat\theta]{{y_H}}{ r}}_{z=0}$ or a corresponding standard
%error
.
When regressing $Y_{H}$ on $R$, include an additive
contribution from $Z$, so that $\mu_{\beta}(R) =\beta_{0} +
R\beta_{1}$ is replaced with $\mu_{(\beta,\gamma)}(R) =\beta_{0} +
\beta_{1}R + \gamma Z$. With sandwich estimates of
$\text{Cov}\{(\hat{\beta}_{H}, \hat{\gamma}_{H})\}$,
% ?state that in case of OLS with sandwich standard errors,
% taking $e$ to be a partial residual makes \eqref{eq:tedef}
% equivalent to the t statistic on the regression coefficient
%
% <!--the below is a passage that I cut a while back, as opposed to an inline note:-->
%
% just as \eqref{eq:sudef}'s $\mathrm{SE}_{u}^2 \left(
%   \bar{Y}_{Z=1} -  \bar{Y}_{Z=0} \right)$ was seen in
% Section~\ref{sec:randProc} to estimate $\var(\bar{Y}_{Z=1} -
% \bar{Y}_{Z=0}| \mathbf{A})$ just as validly as $\var(\bar{Y}_{Z=1} -
% \bar{Y}_{Z=0}| \mathbf{Z})$, the sandwich estimate is consistent for
% the variance of $\hat\gamma$ with conditioning as discussed in
% Section~\ref{sec:model-eey-c-r}
% \citep[Section~\ref{apnd:requ-forpr-eqref}]{lrdauthors:supp}.
the t-ratio comparing $\hat{\gamma}_{H}$ to
$\text{SE}_{s}(\hat{\gamma}_{H})$ induces a generalized score test \citep{boos1992genscoretest}. Implicitly it is a two-sample
t-statistic with covariance adjustment for $R$, as in \eqref{eq:tedef}: with fitting via OLS,
this correspondence would be exact, as noted in Section~\ref{sec:robust-analys-covar}; with the robust MM estimation we
favor, the correspondence is one of large-sample equivalence
%\citep[Section~\ref{sec:suppl-s-refs}]{lrdauthors:supp}
(Appendix~\ref{sec:suppl-s-refs}).
% Just as in RCTs \eqref{eq:sudef}'s $\mathrm{SE}_{u}\left( \bar{y}_{Z=1} -  \bar{y}_{Z=0} \right)$ is valid either with conditioning on $\mathbf{Z}$
% that leaves $\mathbf{Y}_{T}$ and $\mathbf{Y}_{C}$ unconstrained, or
% with conditioning only on $\mathbf{A}$,
% $\text{SE}_{s}(\hat{\gamma}_{H})$ calculated in this way is consistent for $\var^{1/2}(\hat{\gamma}_{H}\mid A)$.
% With large samples, comparing
% $\hat{\gamma}_{H}/\text{SE}_{s}(\hat{\gamma}_{H})$
% to a central $t$ or standard Normal distribution
%  gives a randomization-based test of $H$ \citep[Section~\ref{apnd:requ-forpr-eqref}]{lrdauthors:supp}.
Iterative testing is facilitated by regressing $\mathbf{y}$ on $\mathbf{r}$ and $\mathbf{z}$
with offset variable $\mathbf{d}\tau_{0}$; then only the offset needs to
be modified to test $H: Y_{T} = Y_{C} +  \tau D_{T}$ for a new $\tau$.

\subsection{Post-Fitting Diagnostics} \label{sec:post-fitt-diagn}
Once the Hodges-Lehmann estimate $\hat{\tau}_{\mathrm{HL}}$ has been found, one
inspects the corresponding regression fit for points of high influence.
Bounded influence regression is helpful here.  Besides making
influential points easier to see in residual plots, this limits
effects of data contamination, as non-conforming influence points are
down-weighted as a result of the fitting process. This down-weighting
is reflected in robustness weights, ranging from 1, for non-discounted
observations, down to 0, for the most anomalous observations.
Plotting % As
% an indicator of robustness to non-constant effects.
% (But is this overloading the discussion here?)
robustness weights against residuals may expose opportunities to
improve the fit of $\mu_{\beta}(R)$, or of the treatment effect model;
plotting them against $R$ may expose contaminated sub-regions
of $\mathcal{W}$ that specification testing failed to remove.




\section{Review of selected methods}\label{sec:revi-select-meth}


\subsection{Neyman-type inference about the CACE in RCTs}\label{sec:randProc}

With moderate or large samples, the approach deriving from Neyman
\citeyearpar{neyman:1935} gives randomization- but not
permutation-based inferences, enabling it to test ``weak'' null
hypotheses, for example $H: \EE(Y_T - Y_C) =c$ or
$\EE(Y_{T} - Y_{C}| D_{T}=1) =c$, as opposed to Fisherian ``strong''
hypotheses of form $H: Y_T - Y_C \equiv c$, as tested in
Section~\ref{sec:using-eqref-test}, or $H: Y_T - Y_C \equiv D_{T}c$.

To do this it conditions somewhat differently, on
centered potential outcomes
$\tilde{\mathbf{Y}}_{T} = \mathbf{Y}_{T} - \bar{Y}_{T}$ and
$\tilde{\mathbf{Y}}_C$, as opposed to the uncentered
and $\mathbf{Y}_{C}$.
This centering is around study population means,
$\bar{Y}_{T} = n^{-1}\sum_{i}Y_{Ti}$ and
$\bar{Y}_{C} = n^{-1}\sum_{i}Y_{Ci}$, as opposed to within-group means
$\overline{(Y_{T})}_{Z=1} = N_{1}^{-1}\sum_{i}
Y_{Ti}\indicator{Z_{i}=1}$
and
$\overline{(Y_{C})}_{Z=0} = N_{0}^{-1}\sum_{i}
Y_{Ci}\indicator{Z_{i}=0}$.
As a result, the Neyman-type conditioning variables $\tilde{\mathbf{Y}}_{T}$ and
$\tilde{\mathbf{Y}}_{C}$ are entirely unobserved, in contrast to the
Fisherian conditioning variable $\mathbf{Y}_{C}$, which was at least
partly observed.  But by the same token, for inference about the parameter
$\EE(Y_{T} - Y_{C})$ under a
location-scale family model of $\PP[(Y_{C}, Y_{T})]$,
 %, or $\EE(Y_{T} - Y_{C} | D_{T}=1)$,
the variable
$(\tilde{\mathbf{Y}}_{C}, \tilde{\mathbf{Y}}_{T})$ is uninformative.
Formally, it is S-ancillary \citep{cox2006pos,lehmannRomano2006TSH}
to $\EE(Y_{T} - Y_{C})$. Similarly,  under a location-scale model for $\PP[ (Y_{C}, Y_{T}) |
D_{T}=1]$, $(\tilde{\mathbf{Y}}_{C}, \tilde{\mathbf{Y}}_{T}, \mathbf{D}_{T})$ is S-ancillary for
inference about the TOTE, as is
%$(\tilde{\mathbf{Y}}_{C}, \tilde{\mathbf{Y}}_{T}, \mathbf{D}_{T}, \mathbf{D}_{C})$ and
$(\tilde{\mathbf{Y}}_{C},
\tilde{\mathbf{Y}}_{T}, \mathbf{D}_{T}, \mathbf{D}_{C}, N_{1},
N_{0})$.  The latter is denoted ``$\mathbf{A}^{\dagger}$'' (the symbol ``$\mathbf{A}^{*}$'' having been used in \S~\ref{sec:using-eqref-test} and ``$\mathbf{A}$'' being reserved for a statistic to be
introduced in \S~\ref{sec:model-eey-c-r}).

The archetypical
Neyman-type test statistic \citeyearpar{neyman:1923,neyman1934stratifiedsampling} is
  % If $\tilde{\mathbf{y}}_H$ denotes the potential response
  % $\mathbf{y}_{c}$ that would be entailed by a strong null $H$, in
  % combination with observations $\mathbf{y}$ and $\mathbf{d}$, then
\begin{align} \label{eq:tudef}
t_{u}({\mathbf{y}},\mathbf{z}) &=
\frac{\bar{{y}}_{z=1} -
                             \bar{{y}}_{z=0}}{
\mathrm{SE}_{u} \left\{ \bar{{y}}_{Z=1} -  \bar{y}_{Z=0} \right\}
                             },\, \text{where}
\\
%% where $\overline{v}_{z=j} = n_{j}^{-1} \sum_{i: z_{i} = j}
%% v_{i}$, $n_{j} = \sum_{i}\indicator{z_{i} =j}$, for $j = 0, 1$, and
\mathrm{SE}_{u}^2 \left( \bar{y}_{Z=1} -  \bar{y}_{Z=0} \right) &= \frac{s_{z=1}^{2}({\mathbf{y}})}{n_{1}} +
    \frac{s_{z=0}^{2}({\mathbf{y}})}{n_{0}}  \label{eq:sudef}
% \\
%     s_{z=j}^{2}(\mathbf{v}) &= \frac{1}{n_{j}-1}\sum_{i: z_{i} = j}
% (v_{i} - \bar{v}_{z=j})^{2} .\nonumber
% \end{equation}
\end{align}
is the ordinary unpooled two-sample variance. With conditioning on $\mathbf{A}^{\dagger}$ --- as opposed to the
conditioning on $\mathbf{Z}$, and $\mathbf{Z}$ only, relative to which
two-sample $t$-statistics like~\eqref{eq:tudef} are more commonly
understood --- \eqref{eq:sudef} equals the sum of a zero-mean
disturbance and
${s^{2}({\mathbf{Y}_{T}})}/{N_{1}} +
{s^{2}({\mathbf{Y}_{C}})}/{N_{0}}$, an upper bound of the variance, as
conditioned on $\mathbf{A}^{\dagger}$, of
$\overline{(Y_{T})}_{Z=1} - \overline{(Y_{C})}_{Z=0}$.  In consequence,
if $H_{0}: \EE Y_{T} =\EE Y_{C}$ is true then
$t_{u}(\mathbf{Y}, \mathbf{Z})\stackrel{d}{\rightarrow}\mathrm{Normal}(0,v)$,
some $v\leq 1$, conditional on $\mathbf{A}^{\dagger}$; see Appendix~\ref{apnd:RICLT}.%\citep[Section~\ref{apnd:RICLT}]{lrdauthors:supp}.

% Although $t_{u}$ can also be used in \eqref{eq:01} for permutation
% tests of the sharp null $Y_{T} \equiv Y_{C}$
% \citep[e.g.,][]{chung2013exact}, more commonly it is paired with
% the large-sample approximation, to test the weak null $\EE(Y_{T} -
% Y_{C}) =0$.

More generally, for any $c$ we can assess
$t_{u}(\mathbf{Y}-c\mathbf{D}, \mathbf{Z})$ against a standard Normal
or central $t$ distribution to test
$H_{c}: \EE( Y_{T} - cD_{T}) =\EE (Y_{C} - cD_{C})$.
The set \{$c$:
$H_{c}$ is not rejected at level $\alpha$\}, which can be seen to be an interval, is a
$100(1-\alpha)\%$ confidence set for the TOTE; the $c$ solving
$t_{u}(\mathbf{y}-c\mathbf{d}, \mathbf{z}) = 0$, which can be seen to be unique, is a
corresponding point estimate.
Under full compliance, the two-sided 95\% confidence interval induced in this
manner is the familiar
$\bar{y}_{z=1} -\bar{y}_{z=0} \pm 1.96\, \mathrm{SE}_{u}(\bar{y}_{Z=1}
-\bar{y}_{Z=0})$.  If compliance is incomplete, then determining the
confidence interval or set induced by~\eqref{eq:tudef} requires a distinct explicit
test for each of a range of $c$s: by~\eqref{eq:sudef}, if $\mathbf{d}
\neq \mathbf{z}$ then the value
of $\mathrm{SE}_{u}(\overline{(y-c d)}_{Z=1}
-\overline{(y - c d)}_{Z=0})$ depends on $c$.  As compared to Wald-type
confidence intervals, that is intervals of form $\hat\tau \pm q_{*}
\mathrm{SE}(\hat\tau)$ with $\mathrm{SE}(\hat\tau)$ a single,
hypothesis independent quantity, under partial compliance this iterative method
% of separate evaluation of $\mathrm{SE}_{u} \left\{
%   \overline{({y}_{H_{\tau}})}_{Z=1} -
%   \overline{({y}_{H_{\tau}})}_{Z=0} \right\}$ for each $H_{\tau}$,
improves correspondence of nominal and actual confidence levels
% , particularly if the ``instrument,'' $Z$, is ``weak,''
% i.e. $\PP(D_{T} =1) \approx \PP(D_{C} =1)$
\citep[Sec.~7]{imbens:rose:2005,baiocchiChengSmall2014IVtutorial}.

\subsection{Fuzzy RD}