% !TEX TS-program = pdflatex
% !TEX encoding = UTF-8 Unicode

% This is a simple template for a LaTeX document using the "article" class.
% See "book", "report", "letter" for other types of document.

\documentclass[11pt]{article} % use larger type; default would be 10pt

\usepackage[utf8]{inputenc} % set input encoding (not needed with XeLaTeX)

%%% Examples of Article customizations
% These packages are optional, depending whether you want the features they provide.
% See the LaTeX Companion or other references for full information.

%%% PAGE DIMENSIONS
\usepackage{geometry} % to change the page dimensions
\geometry{a4paper} % or letterpaper (US) or a5paper or....
% \geometry{margin=2in} % for example, change the margins to 2 inches all round
% \geometry{landscape} % set up the page for landscape
%   read geometry.pdf for detailed page layout information

\usepackage{graphicx} % support the \includegraphics command and options

% \usepackage[parfill]{parskip} % Activate to begin paragraphs with an empty line rather than an indent

%%% PACKAGES
\usepackage{booktabs} % for much better looking tables
\usepackage{multirow} % for better arrays (eg matrices) in maths
\usepackage{paralist} % very flexible & customisable lists (eg. enumerate/itemize, etc.)
\usepackage{verbatim} % adds environment for commenting out blocks of text & for better verbatim
\usepackage{subfig} % make it possible to include more than one captioned figure/table in a single float
% These packages are all incorporated in the memoir class to one degree or another...

%%% HEADERS & FOOTERS
\usepackage{fancyhdr} % This should be set AFTER setting up the page geometry
\pagestyle{fancy} % options: empty , plain , fancy
\renewcommand{\headrulewidth}{0pt} % customise the layout...
\lhead{}\chead{}\rhead{}
\lfoot{}\cfoot{\thepage}\rfoot{}

%%% SECTION TITLE APPEARANCE
\usepackage{sectsty}
\allsectionsfont{\sffamily\mdseries\upshape} % (See the fntguide.pdf for font help)
% (This matches ConTeXt defaults)

%%% ToC (table of contents) APPEARANCE
\usepackage[nottoc,notlof,notlot]{tocbibind} % Put the bibliography in the ToC
\usepackage[titles,subfigure]{tocloft} % Alter the style of the Table of Contents
\renewcommand{\cftsecfont}{\rmfamily\mdseries\upshape}
\renewcommand{\cftsecpagefont}{\rmfamily\mdseries\upshape} % No bold!

%%% END Article customizations

%%% The "real" document content comes below...

\title{Brief Article}
\author{The Author}
%\date{} % Activate to display a given date or no date (if empty),
         % otherwise the current date is printed 

\begin{document}
\begin{table}
\footnotesize
\begin{tabular}{cc|ccc|ccc|ccc}
\hline

&& \multicolumn{ 3 }{c}{Permutation}&\multicolumn{ 3 }{c}{``Limitless''}&\multicolumn{ 3 }{c}{Local OLS}\\
$n$& Error & Bias&Coverage&Width&Bias&Coverage&Width&Bias&Coverage&Width \\
\hline 
\hline 
\multirow{2}{*}{ 50 } & $\mathcal{N}(0,1)$ &0.25&0.87&1.18&0.01&0.93&2.3&0&0.93&2.24 \\ 
 & $t_3$ &0.25&0.9&1.49&0&0.95&2.89&0&0.95&3.6 \\ 
\hline 
\multirow{2}{*}{ 250 } & $\mathcal{N}(0,1)$ &0.25&0.51&0.51&0&0.94&1.02&0&0.95&0.99 \\ 
 & $t_3$ &0.25&0.64&0.63&0&0.95&1.25&0.01&0.95&1.67 \\ 
\hline 
\multirow{2}{*}{ 2500 } & $\mathcal{N}(0,1)$ &0.25&0&0.16&0&0.95&0.32&0&0.95&0.31 \\ 
 & $t_3$ &0.25&0&0.2&0&0.95&0.39&0&0.95&0.54 \\ 
\hline
\end{tabular}
\caption{Effect=0}
\end{table}



\begin{table}
\footnotesize
\begin{tabular}{cc|ccc|ccc|ccc}
\hline

&& \multicolumn{ 3 }{c}{Permutation}&\multicolumn{ 3 }{c}{``Limitless''}&\multicolumn{ 3 }{c}{Local OLS}\\
$n$& Error & Bias&Coverage&Width&Bias&Coverage&Width&Bias&Coverage&Width \\
\hline 
\hline 
\multirow{2}{*}{ 50 } & $\mathcal{N}(0,1)$ &0.25&0.86&1.18&-0.01&0.94&2.29&-0.01&0.93&2.24 \\ 
 & $t_3$ &0.24&0.9&1.49&-0.02&0.94&2.87&-0.03&0.95&3.57 \\ 
\hline 
\multirow{2}{*}{ 250 } & $\mathcal{N}(0,1)$ &0.25&0.52&0.51&0&0.95&1.02&0&0.95&1 \\ 
 & $t_3$ &0.25&0.65&0.63&0&0.95&1.25&0&0.95&1.66 \\ 
\hline 
\multirow{2}{*}{ 2500 } & $\mathcal{N}(0,1)$ &0.25&0&0.16&0&0.95&0.32&0&0.95&0.31 \\ 
 & $t_3$ &0.25&0&0.2&0&0.95&0.39&0&0.95&0.54 \\ 
\hline
\end{tabular}
\caption{Effect=0.2}
\end{table}

\end{document}
